%\pagestyle{headings}
%example of the text

\section*{Introduction}
\addcontentsline{toc}{section}{Introduction}
Peer-to-peer (P2P) lending is a modern way of money lending that was first introduced by the British platform Zopa in 2005. As the term Peer-to-peer suggests (peers are people of the same status), individual people are on both sides of the lending process. 

Over the years, P2P lending has shown that it can compete with clasical financial institutions, providing less costly products. The economic succes of P2P lending is being followed by an extensive amount of economic research focusing on this topic. 

Part of the research focuses on P2P lending from the perspective of the borrowers. \cite{Livingston2012} 
examines the possibility of substituting the payday loans by P2P lending. It's argued that some consumers may benefit from substitution, while others may find the characteristics of P2P loans (funding speed, access, required credit score) unfeasible. \cite{deRoure2016} 
use the data on P2P lending by Auxmoney and banking by the Deutsche Bundesbank to examine whether P2P lending is a substitute or complement to bank lending from the viewpoint of borrower. It is concluded that although the interest rates are comparable when controlled for the level of risk, loans channelled through P2P platforms are riskier with higher interest rate than bank loans.

Another part focuses on the investors' perspective. Some papers examine the effect of soft characteristics on the credibility of the borrower. Among these, \cite{Herzenstein2011} find evidence of strategic herding behaviour as well as evidence of herding being beneficial to the investors on the platform Prosper. \cite{Duarte2012} also use the data from Prosper to examine trustworthiness of the borrower based on their picture. \cite{Lin2013} find that the investors reasonably use friendships as signal of credit quality, again based on the data from Prosper.

Other papers focusing on the investors' perspective provide investment tools that estimate the risk of default \cite[]{Mild2015} or optimize the investments to provide portfolio of required characteristics \cite[]{Guo2016}. \cite{Hudcova2017} studies the investors’ risk behaviour and finds that investors on P2P lending platform Zonky behave riskier than their risk aversion profile would suggest.

This thesis tries to contribute to the economic knowledge about P2P lending by investigating how investments in P2P loans fit into the framework of portfolio investment. Specifically, it is examined in what relationship P2P lending is with other types of investment products. According to the best of my knowledge, this topic has not been examined yet, academic work has only focused on P2P lending as substitute to bank loans from the perspective of borrower up to date.

The rest of the thesis is organised in the following way. In the first chapter, \textit{Overview of P2P lending} is presented, providing definition, describing the possible models, comparing P2P lending with other investment products and showing its importance. The second chapter presents the \textit{Czech market for P2P lending}. Chapter three provides discussion of \textit{Methodology and conceptual framework}, with the following focusing on \textit{Data collection and empirical work}. At the end, the thesis is concluded.

\newpage
\section{Overview of P2P lending}
Although many people intuitively understand the term Peer-to-Peer Lending and its mechanisms, there exists no generally agreed upon definition of it that would fit all the uses. In this chapter, I provide the definition that generally fits the use of the term, show both how the platforms work generally and what the most common differences in platform processes are, compare P2P lending to conventional money lending products and describe the potential importance of P2P lending in economy.

\subsection{Definition}

In this thesis, Peer-To-Peer Lending is understood to be “the loan origination process between private individuals on online platforms where financial institutions operate only as intermediates”. The definition is based on the one from \cite{Bachmann2011}, only the intermediaries are not necessarily required by law.

As mentioned earlier, the definitions vary. For instance Zopa, the first P2P-lending platform worldwide, describes itself this way: “…we directly match people looking for a low rate loan with investors looking for a higher rate of return.”\footnote{\url{https://www.zopa.com/about/how-zopa-works}, (accesed January 3, 2019)}

The Peer-To-Peer Finance Association, established in 2011 as a representative and self-regulatory body for peer-to-peer lending in the United Kingdom defines P2P Lending this way: “Peer-to-Peer lending enables investors to lend funds directly to borrowers via an online platform. Retail investors access a platform to provide loans to consumers or small business borrowers.  Whilst platforms facilitate the lending, undertake credit assessments and other risk management, they do not act as a counter-party to the loan, and contracts are direct between the investor and the borrower.”\footnote{\url{https://www.p2pfa.org.uk/}, (accesed January 3, 2019)}

It may seem that the definitions do not differ significantly. The point is, that both the investor and borrower should be private individuals. If the investor were a company, the concept would be fundamentally no different to common bank loans. On the other hand, if the borrower is a company, there is no difference compared to a corporate bond.

\subsection{How P2P lending works}

P2P lending is a phenomenon that has various forms \cite[]{Bachmann2011, Chaffee2012, Fong2015}. As \cite{Kirby2014} states, the business models of P2P lending platforms are diverse, depending both on the regulatory environment and the principles of the platforms. These may include the process of interest rate determination, the fees charged, the process of clients' credibility verification and the amount of borrower information provided to the investors \cite[]{Bachmann2011}.

Despite all the possible differences in practical functioning of the platforms I will try to describe the possible operational models in this subsection.

\subsubsection{P2P lending as part of a wider context}

As \cite{Herrero-Lopez2009} states, P2P lending is a phenomenon that merges the old idea of personal credits with the technology of internet. In other words, the new thing about P2P lending platforms is their online dimension \cite[]{Chaffee2012}. The connection than significantly reduces both the transaction costs and the costs of finding a fitting counterpart to the lending/borrowing process \cite[]{Chaffee2012, Kirby2014}.

P2P lending platforms are one type of the finance platforms. Others are donation crowdfunding platforms, reward crowdfunding platforms, and equity crowdfunding platforms \cite[]{Pierrakis2013}.

\cite{Pierrakis2013} states that in P2P lending the primary motivation of the funder is financial, as the loan is to be repaid with interest. This claim is contradicted by both \cite{Ashta2009} and \cite{Bachmann2011}. \cite{Bachmann2011} states that the basic division of P2P lending platforms is based on its' commerciality or non-commerciality, where money on non-commercial platforms are rather donated. The fact is that, although non-commercial P2P lending platforms do exist (the most well known is kiva.com), the loan is always to be repaid, even though there is no interest provided to the lender \cite[]{Ashta2009}. If the money were not to be repaid, the operation would be considered a (donation) crowdfunding \cite[]{Pierrakis2013}.

As mentioned by \cite{Pierrakis2013}, the P2P lending model has also been applied to lend money to companies. This fact often causes confusion in the definitions and the use of terms. Often we can find terms such as peer-to-peer business lending \cite[]{Ziegler2018}, that do not make much sense. As both \cite{Chaffee2012} and the definition of P2P lending used in this thesis state, in P2P lending, individuals stand on both sides of the transaction. Once the borrower is a business entity, the process should be called a peer-to-business (P2B) lending \cite[]{ecrowd!2017}. 

\subsubsection{Legal aspects of P2P lending}

The P2P lending platforms are significantly affected by the laws of the area where they want to provide their services and by the domestic regulation. The laws and regulations provide the basic rules of the game, affecting the possibilities and therefore the outcome. In other words, a business model that works very well in one country may be illegal in others, making it impossible to use.

In the given context, \cite{Kirby2014} identifies 5 models of P2P lending regulation. These are \textit{Exempt or unregulated through lack of definition}, \textit{Regulated as an intermediary}, \textit{Regulated as banking}, \textit{The US model} and \textit{Prohibited} \cite[For more details see][]{Kirby2014}.

Within a given legal environment, different business models may be adopted by the platforms. Three such models are identified by \cite{Kirby2014}. Although the examples are far from being exhaustive, they are used to show the possible diversity from both practical and legal poin of view.

In the \textbf{Client segregated account model}, the platform acts as a pure intermediary \cite[]{Fong2015}. The funds from both lenders and borrowers are separated from the platform balance sheet, therefore the contract still applies in the case of platform going out of business \cite[]{Kirby2014}.

In the \textbf{Notary model} the loans are not made directly to the borrowers by the investors. The platform is affiliated with a bank which is the issuer of the loan. The bank sells the loan to the platform and a note representing the contributed value is issued by the platform to the investor with return depending on the repayment of the original loan. The investment is therefore made into a note, not the loan itself. This model is very often used in USA. \cite[]{Chaffee2012, Fong2015, Kirby2014} In the case of the platform going bankrupt the contract may be affected.

Under the \textbf{Guaranteed return model} the lenders can invest through the platform at a rate of return guaranteed by the platform. The platform must provide the guaranteed return or must end its activity. This model is known mainly from China. \cite[]{Kirby2014}

The differences in the legal models are important because they affect the actual regulation affecting the particular platform. In other words, as platforms vary in their models, the legal requirements on them vary too. This means that two P2P lending platforms within the same legal environment may be regulated differently. \cite[]{Jorgensen2018}

\subsubsection{The process of interest rate determination}

The platforms may also differ in the interest rate determination process \cite[]{Bachmann2011}. \cite{Vysusil2016} and \cite{wiseclerk2013} identify three models of interest rate determination.

The \textbf{Auction based determination of interest rate} is used by platforms such as Monestro.com. Although both \cite{Vysusil2016} and \cite{Bachmann2011} provide Prosper as an example of a portal that uses this process, this is not true anymore and was not even at the time of publishing their papers as Prosper changed their model on December 19, 2010 and uses a different model nowadays \cite[]{Renton2012}.

In this model the borrowers state the highest interest rate they would be willing to accept. The lenders than bid on given loans to provide the borrower with lower interest rate offer. The lenders who offer the lowest interest rates are than part of the final deal. \cite[]{Bachmann2011, Vysusil2016, wiseclerk2013}

As \cite{wiseclerk2013} mentions, this model can be applied in two different ways. Either the interest rates are \textbf{Uniform} or \textbf{Mixed}. In the case of uniform interest rate, the interest rate of all succesful lenders is set at the rate of the highest succesful bid. On the others hand, in the case of mixed interest rates each succesful lender gets his lowest bid. \cite[]{wiseclerk2013}

In the \textbf{Calculation based determination of interest rate}, the interest rate is calculated based on the loan and borrower characteristics by the platform \cite[]{Bachmann2011, Vysusil2016, wiseclerk2013}. Many platforms use this interest rate determination process, such as LendingClub, Zopa and nowadays Prosper, that formerly used the auction system.

One further model of interest rate determination is identified by \cite{Vysusil2016} and \cite{wiseclerk2013} , called the \textbf{Borrower set interest rate}. As its name suggests, the borrower sets his interest rate, with the only limitation being the willingness of the lenders to invest in a loan with given conditions \cite[]{Vysusil2016, wiseclerk2013}. An example of this model is the estonian platform Bondora.

\subsubsection{Lenders' possibilities of investment selection}

Another distinction characterised by \cite{wiseclerk2013} is the level of selection possibilities given to the investor by the platform.

Either the loans are \textbf{Individually listed} or the investments are made into the \textbf{Markets}. 

In the first case, the investors choose from a list of individual loans placed on a marketplace. On the other hand, in the second case, it is only possible to invest into the whole market of loans with selected criteria, but the individual loan can not be chosen. \cite[]{wiseclerk2013}

\subsubsection{Other variable characteristics of P2P lending platforms}

The platforms may further vary in virtually infinite amount of numerical parameters, including the required level of borrower credibility, the borrower information provided to the investors, the minimum amount of money invested, other services provided by the platform such as insurance or the level of fees charged by the platform for the services \cite[]{Ashta2009, Bachmann2011, Fong2015}. When platforms receive fees based on the performance of the loan, the interest of the platform is aligned with the interest of the lender \cite[]{Fong2015}, which is often true.

\subsection{Conventional Investment Products}

This thesis tries to examine investment in P2P loans in context of conventional investment products. Therefor this subsection should provide a short overview of conventional investment products. For a deeper analysis see \cite{Reilly2000} or \cite{Zigraiova2010}.

\cite{Reilly2000} identify 7 cathegories of portfolio investments. These are \textit{Fixed income investments}, \textit{Equity instruments}, \textit{Options}, \textit{Futures contracts}, \textit{Investment companies}, \textit{Real estate} and \textit{Low liquidity investment}. \cite{Zigraiova2010} adds one more cathegory, the \textit{Comodities}. Another investment option are the foreign currencies and recently also the cryptocurrencies.

The \textbf{fixed income investments} have a contractually mandated payment schedule. Specific payments at predetermined times are promised by the contract with the investors being actually lenders to the issuers. The examples are Savings accounts, Certificates of deposit, Capital market instruments or Prefered stock. \cite[]{Reilly2000}

The \textbf{equity instruments} provide returns that are not agreed upon in advance. The ownership of a company is represented by the instruments. \cite[]{Reilly2000} 

\textbf{Options} provide the owners rights to trade common stock for a given price and time period \cite[]{Reilly2000}.

The \textbf{futures contracts} are contracts on an exchange of an asset for a given price at a delivery date \cite[]{Reilly2000}. Compared to options, futures do not only provide rights to perform a transaction but also a commitment.

The \textbf{Investment companies}, also called \textbf{mutual funds}, are companies that sell their own shares and use the proceeds to buy investment instruments. The ownership of the fund's shares than provides indirect ownership of an investment portfolio. \cite[]{Reilly2000}

Although \textbf{Real estate} investments are often thought to be accessible only for investors with high level of capital, investment products that allow low-capital investors to invest in real estate exist. Such product is an Real estate investment trust. Other investment possibilites are Direct real estate investment, Land development or Rental property investment. \cite[]{Reilly2000}

\textbf{Low liquidity} investments are not traded on a centralised market as they are not standardised products. This group includes Antiques, Arts, Coins and stamps or Diamonds. \cite[]{Reilly2000}

\textbf{Commodities} are standardized products or materials. They may include metals, raw materials, or agricultural products and are traded on centralised markets.

\textbf{Foreign currencies} and \textbf{cryptocurrencies} are latelly traded mainly as risky investment products providing possibly high rates of return.

\subsubsection{Comparison with P2P lending}

As investments in P2P loans provide contractually binding payments, they are a type of fixed income investments. The P2P loans generally provide higher rate of return with the cost of higher risk than most of the other fixed income investments. The reason is that there is no guarantee of repayment provided by the government (as in the case of government securities or most of the savings accounts and certificates of deposits). The comparison with further fixed income investments depends mainly on the specific characteristics of given loans. The liquidity of the P2P loans depends highly on the P2P platform used to issue the loan and its rules. Therefore, from the theoretical point of view the P2P loans could be understood as having similar characteristics as corporate bonds.

Under the assumption of P2P loans having similar characteristics as corporate bonds, the other investments should provide higher returns at the cost of higher risk \cite[]{Reilly2000}.

Therefore the P2P loans are expected to be substitutes to corporate bonds.

\subsection{Importance of P2P lending}

Before further advancing in the thesis, it should be advocated that P2P lending was selected as the main topic of the thesis. 

From the theoretical point of view, as the minimum lending amount is very low, small savers are allowed to invest which may provide them better conditions than bank accounts would \cite[]{Ashta2009}. Further interesting features are discussed and divided into benefits and risks by \cite{Kirby2014}. The benefits are 

\begin{itemize}
\item Credit flows are created or increased, supporting the economic growth.
\item P2P lending fills a gap left by banks.
\item Higher cost efficiency thanks to the use of technologies.
\item Provides lower cost of capital and higher returns thanks to lower costs.
\item Investment diversification.
\item Convenience for the investors.
\item Increased competition that provides benefits to borrowers, lenders and the whole economy.
\end{itemize}

The risks, on the other hand include

\begin{itemize}
\item Risk of default
\item Platform risk – risk of platform being permanently or temporarily shut down
\item Risk of fraud
\item Information asymmetry and quality
\item Risk of investor inexperience
\item Liquidity risk
\item Risk of cyber attack
\end{itemize}

Since 2005, when the first P2P lending platform Zopa was founded, P2P lending has showed that it is also important practically. The biggest platform in the USA, Lending Club, has issued loans worth over 44 billion USD since its foundation, with 10,9 billion issued only in 2018 \cite[]{LendingClub}.
For comparison, according to the Federal Reserve Bank of USA, the amount of all consumer loans originated in 2018 was 17,6 trillion  \cite[]{FRED}. 
The most recent data on the European market excluding the UK show that over the period 2013 – 2016 the volume of issued P2P consumer loans more than quadrupled from 157 million euro to 697 million euro \cite[]{Ziegler2018}. 

In the Czech environment, the biggest P2P lending platform Zonky accounted for over 3\% of the Czech market for consumer loans in 2018. In nominal terms the result of 3,2 billion CZK meant a growth of more than 100\% year-on-year. %\cite[]{ctk_2019}

To conclude, P2P lending is a modern way of lending money that showed it is able to provide competitive products to both borrowers and lenders. Its market share steadily grows all over the world. Many features that are theoretically important have been identified providing justification for analysing this phenomenon economically.
\newpage
\section{Czech market for P2P lending}

In this section the Czech P2P lending market is examined. The list of P2P lending platforms focusing on Czech market was made using the web page \cite{p2pforum}. In the discussion thread with the topic "P2P PŮJČKY V ČR (P2P lending in the Czech Republic)", 22 platforms were indentified. Out of these, 6 were found to be rather P2B lending platforms (Roger, Symcredit, Zalep.to, Pujcmefirme, Fingood and Spotipay), two were found not to be active anymore (FerratumP2P and Benefi), five were not found neither at their reported web page nor by the google search (Snapo, Finx, Intereson, Loanis and Ratecash), one is rather a platform for developer projects investment (Upvest) and one is rather a discussion platform (Investice-pujcka). These 15 platforms were not examined further, leaving only 7 active P2P lending platforms to be examined. The focus of the research was on the topics discussed in the subsection "How P2P lending works" to get more standardized outcome. 



\textbf{Zonky} is the most well-known P2P lending platform in the Czech Republic. It uses the notary model with the single difference that the platform is also taking the place of the loan issuer \cite[]{ZonkyOPParticipace, ZonkySmlouvaHotUv}. The note is called "Participace" (Participation) \cite[]{ZonkyOPParticipace}. The calculation based determination of interest rate is used \cite[]{ZonkyFAQ}, the loans are individually listed \cite[]{ZonkyTrziste} and high borrower credibility is required (currently the interest rates range from 3,99\% to 19,99\%, 2,99\% is planned)\cite[]{ZonkyFAQ}. The information provided to the investor is the length of the loan, its rating cathegory, the source of borrowers income, the region of the borrower's residence, and the verification used by Zonky \cite[]{ZonkyTrziste}. The minimum investment nowadays is 200CZK, 100 is being tested \cite[]{ZonkyFAQI}. The loan can be insured by the borrower against insolvency \cite[]{ZonkyPoj}. The fees differ across the ratings, less risky participations have lower fees with the range being 0,2-5\% p.a. of the outstanding principal \cite[]{ZonkySazI}. Once the loan is delayed for 36 days or more, the fee is not paid \cite[]{ZonkyFAQI}. 30\% of the money judiciary enforced is charged by Zonky \cite[]{ZonkySazI}. The Participations can be sold on secondary market, if there are no problems with repayment of the original loan \cite[]{ZonkyParametryST}. 1,5\% of the outstanding participation is charged as a fee when the participation is sold. The fee is not charged if the participation was held for more than 12 months \cite[]{ZonkySazI}. The price is set by the platform at the level of the principal outstanding \cite[]{ZonkyFAQI}.

Zonky gives legal entities the opportunity to invest through the platform as well, becoming an institutional investor \cite[]{ZonkyFAQI}.

The platform \textbf{Bondster} uses the client segregated account model, where the claim is bought from a third party loan issuer using the platform as an intermediary \cite[]{BondsterSmlouvaPostPoh}. The interest rate is determined by the issuers using the calculation based process \cite[]{BondsterSmlouvaPostPoh}. The loans are individually listed \cite[]{BondsterTrziste} and the borrower credibility required depends on the loan issuer.  Some of these provide loans with interest rate up to 600\% p.a. \cite[]{BondsterPoskytovatele}. The investments start at 100Kč \cite[]{BondsterFAQ}. The loans may be secured by a collateral, have the guarantee of payback provided by the issuer of the loan when the loan performs badly \cite[]{BondsterZajisteni}, or have the possibility of selling the claim back to the issuer under predetermined conditions \cite[]{BondsterZajisteni, BondsterPodminkyVystoupeni}. The fees charged are 1\% p.a. of the invested value counted each day, the performance of the loan does not affect the fees \cite[]{BondsterSazInv}. The information provided to the investor includes  the issuer identification, basic characteristics of the loan, the loan to value of collateral ration, information on the buyback guarantee offer, length of the loan, the source of the borrower income and the borrower region \cite[]{BondsterTrziste}.

Some of the borrowers are legal entities \cite[]{BondsterTrziste}. Therefore the platform is both P2P and P2B lending platform. 

Another platform on the Czech P2P lending market is called \textbf{Prestito}. The platform does not provide the service of intermediation, it only connects borrowers with lenders, meaning that, except for the connection, everything depends on the respective individuals and their agreement \cite[]{Prestito}. The interest rate is set using the auction model, although the interest rate may not be the only determinant as investors may make offers with parameters very different from those that the borrower wanted. This means that borrower may ask for a loan without a collateral but may be offered loan with collateral for much lower interest rate. The borrower may than choose any offer \cite[]{PrestitoFAQInv}. The loans are individually listed \cite[]{PrestitoTrziste}, with minimum investment of 5000 Czech crowns \cite[]{PrestitoFAQInv}. There is no level of borrower credibility required by the platform \cite[]{PrestitoFAQInv}. The borrowers choose what information they want to provide, and the investors may ask for further information. \cite[]{PrestitoFAQInv}. The information on fees varies (there are three pages that provide information about fees and contradict each other), but there is no fee for servicing the loan as the platform does not provide this service, the fees are one time payments only \cite[]{PrestitoPoplatky1, PrestitoPoplatky2, PrestitoPoplatky3}.  Mainly legal services are provided by the platform for fees \cite[]{Prestito}.

The Lithuanian P2P lending platform \textbf{FinBee} entered the Czech market with its separate platform for Czech clients. The platform uses client segregated account model \cite[]{FinbeeDohoda}. The calculation based determination of interest rate is used \cite[]{FinbeeDohoda}, although the platform is ready for the auction system \cite[]{FinbeeTrziste}. The loans are individually listed \cite[]{FinbeeTrziste} and quite low borrower credibility is required as loans with interest rates up to 33\% p.a. are issued \cite[]{FinbeeFAQ}. A lot of information is provided to the lenders: loan term and purpose, age, gender, the post town, marital status, number of dependents, residential status, education, occupation, number of months at current employer, total amount of years working, monthly income, existing loans payments, employment status, outstanding debts and financial obligations and the date of the last registered payment of debt and financial obligations \cite[]{FinbeeTrziste}. The minimal investment amounts to 250 Czech crowns \cite[]{FinbeeDohoda}. No further services were found to be provided by the platform. No fees are charged for the investors except for the case of borrowers default, where the costs of legal enforcement are charged to the lender \cite[]{FinbeeDohoda}. The fees charged to the borrower depend on the rating \cite[]{FinbeeFAQ}. The loans can be sold on secondary market for any price \cite[]{FinbeeDohoda}.

The Slovak platform \textbf{Žlutý meloun} entered the Czech market in February 2016 and merged it with the Slovak market as both Czech and Slovak investors can lend money to both Czech and Slovak borrowers \cite[]{ZlutymelounONas}. Various products are provided by the platform, with varying characteristics \cite[]{ZlutymelounProdukty}. For the Czech loans the notary model is used, in the case of the Slovak loans the client segregated account model is used \cite[]{ZlutymelounBezpecnost}. Both auction and calculation based determination of interest rate is used, each is used for a different type of loan (the calculation based determination is used in the case of refinancing and for secured loans) \cite[]{ZlutymelounProdukty}. The loans are individually listed, the minimum investment is 500CZK \cite[]{ZlutymelounInvestovani} and quite low credit rating is required by the platform as the highest interest rates are over 30\% p.a. \cite[]{ZlutymelounStatistika}. The fees charged by Žlutý meloun to investors are 1\% of the installments paid by the borrowers. If the installments are not paid, the fees are not charged. Exceptions are in the case of the purpose loans (Cash Free loans) where the fees are set at the level of 0,33\% \cite[]{ZlutymelounPoplatky}. Selling the claim on the secondary market is charged 1,5\% of the claim by the platform \cite[]{ZlutymelounSazebnik}. The information provided to the investor is rating, country, level of verification, maturity of the loan, recommended interest rate, the use of the loan, the insurance and the securization of the loan \cite[]{ZlutymelounTrziste}.

The platform \textbf{Banking Online} uses the Client segregated account model \cite[]{BankingOnlineSmlouva}. The interest rates are determined based on the auction process \cite[]{BankingOnlineJakToFunguje}, and are mixed \cite[]{BankingOnlineZadatele}. The minimum investment into the individually listed loans is 1000CZK \cite[]{BankingOnlineJakInvestovat}. 0,8\% of each payment to the lender is charged by the platform and fees are also charged for provision of additional documents (flat rate 60CZK + 50CZK per document) \cite[]{BankingOnlineSazebnik}. The platform performs as a pure intermediary and the credibility requirements are set entirely by the lenders.

The last platform currently active at the Czech market is called \textbf{Bankerat}. On this platform the loans are individually listed with the interest rate being determined in the auction \cite[]{BankeratTrziste}. The rates are mixed \cite[]{BankeratJakToFunguje}. The information provided is securization, maturity, age, gender, marital status, number of adult household members, number of dependents, kind of housing, ownership form, region, source of income, level of the monthly income, monthly household costs, total amount of other liabilities, information on seizure and trials and the history at the platform \cite[]{BankeratTrziste}. The platform offers 5 levels of loans based on the official confirmatios or collateral required, the maximal amount of money offered and the maturity of the loan. Up to 10 000CZK for up to 1 year can be borrowed without any confirmation, and a loan for up to 600 000CZK with maximum maturity of 6 years can be issued when a real estate collateral is provided \cite[]{BankeratPujcka}. Client segregated account model is used \cite[]{BankeratSmlouva}. The investor fee is 1\% of the nondefault outstanding principal each year \cite[]{BankeratPoplatky}. Information on minimal investment is not provided. 

\newpage
\section{Methodology and conceptual framework}

\subsection{Unfolding brackets method}

Unfolding brackets or staircase method is a method that estimates a respondents' characteristic based on questionaire survey \cite[]{Falk2016}. In this thesis, it is used to estimate the risk aversion profile as suggested by \cite{Falk2016}. It consists of a sequence of lotery choices, meaning that each time the question is of the form:

\textit{"What would you prefer: a 50 percent chance of winning Y Euro when at the same time there is 50 percent chance of winning nothing, or would you rather have the amount of X Euro as a sure payment?"}

Although X and Y are provided by \cite{Falk2016}, it may be chosen arbitrarily by the researcher. In the case that sure payment is chosen, X decreases for the next question. On the other hand, when lotery is selected, X increases for the next question.

In fact the idea of this method is in finding the value of X for which the respondent would be undecisive if she were asked the question. The higher this value, the less risk averse the respondent should be (at least in the domain of finance). The problem is that finding this value directly is problematic. We could ask the respondents what the value would be, but such question could be often misunderstood.

This method finds the interval in which the value lies. With each question the possible interval is divided into two parts (not necessarily of the same range). By answering the question the researcher is informed in which interval the value actually lies. Generally, the more iterations of the lotery question are asked, the more precise the result is as the whole range is divided into more intervals. On the other hand, it is good to have fewer iterations for the sake of simplicity and not to iritate the respondents too much. 

To account for both of the above mentioned problems, the range of the intervals is not the same for each risk aversion cathegory in this thesis, which is in contrast to the use of both \cite{Hudcova2017} and \cite{Falk2016}. This is to provide more precise results around the risk neutral profile. The reason is that most of the investors are expected to be close to risk neutral and that the same nominal difference is more interesting around the risk neutral profile. Using only three iterations of the method, the investors are divided into eight cathegories based on their risk aversion profile.








\newpage
\section*{Conclusion}
\addcontentsline{toc}{section}{Conclusion}

This is the best place for you to summarize your remarkable
achievements you write about in your thesis.
