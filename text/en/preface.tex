\chapter*{Introduction}
\addcontentsline{toc}{chapter}{Introduction}

This work aims to improve selected \gls{nlp} tasks for Czech with the use of recently published \gls{sota} techniques. 
 
In famous Deep Learning book \citep{Goodfellow-et-al-2016}, \gls{nlp} is defined as "...the use of human languages, such as English or French, by a computer.". \gls{nlp} brings a variety of problems to solve from oral--written language conversion, machine translation, syntax analysis used for automatic grammar correction as well as it serve as a base for further linguistic processing. Semantic analysis includes in addition to already mentioned machine translation other tasks like sentiment analysis, natural language text generation or recognition of homonymy or polysemy of given words. It could solve sophisticated assignments as answering questions about the input text document.
\par

This work applies the most successful \gls{nlp} methods (transfer learning of multilingual bidirectional language model) of recent years to Czech \gls{nlp} task. Such multilingual models are trained on a big bunch of data in different languages. In the case of applied models, one of training languages is Czech, although it forms only a small part of the data. %TODO procenta a zdroj
Selected tasks are tagging, lemmatization and sentiment analysis. Tagging and lemmatization represent syntax analysis, in contrast with sentiment analysis which is a part of semantic task. %TODO dát priklady jinych tasku
Task were chosen from both semantics and syntax to show how pre-trained multilingual language models can help with different types of \gls{nlp} tasks. This work builds on previous work on tagging and lemmatization contextualized embeddings \citep{straka2019czech} and aims to state new \gls{sota} results at least for syntactic tasks.
\par
This work is divided into five chapters. Theoretical background in \gls{nlp} and used \gls{ai} methods and related work is provided in the \hyperref[chap:theandme]{following chapter}.
Implementation documentation in chapter \ref{chap:impl} is followed by discussion about used methods and experiment results in chapter \ref{chap:diss}. Result models are accessible to user exploration as described in chapter \ref{chap:userdoc} and text is closed by \hyperref[chap:concl]{conclusion} with future work proposals.


