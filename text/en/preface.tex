\chapter*{Introduction}
\addcontentsline{toc}{chapter}{Introduction}


This work aim to improve some natural language processing (NLP) tasks for Czech \todo{seznam zkratek} with the use of recently published artificial intelligence (AI) state-of-the-art techniques. In Deep Learning book \cite, NLP is
defined as "...the use of human languages, such as English or French, by a computer." \cite[]{Goodfellow-et-al-2016}. NLP offers a variety of problems to solve from oral--written language conversion, machine translation, syntax analysis used for automatic grammar correction as well as it serve as a base for further linguistic processing. Semantic analysis includes in addition to already mentioned machine translation tasks like sentiment analysis, natural language text generation or recognition
of homonymy or polysemy of given words. It could solve sophisticated assignments as answering questions about the input text document.

\cite{BERT_ORIG}
\section*{Tasks definition}
\todo{obrazky k taskum s priklady}
This work aims to apply the most successful NLP methods of recent years to Czech NLP task - namely tagging, lemmatization and sentiment analysis. Tagging and lemmatization represents syntax analysis, in contrast with sentiment analysis which represents semantic type of tasks. \todo{dát priklady jinych tasku}
The practical part of this work is focused on these tasks:
\begin{itemize}
\item POS tagging \\
\textit{input}: a word \\
\textit{output}: part-of-speech tags -- as noun, pronoun, punctuation mark etc.
\item lemmatization \\
\textit{input:} a word \\
\textit{output:} lemma -- a base form of a given words, meaning for example nominative of singular for nouns or infinitive for verbs. 
\item sentiment analysis \\
\textit{input:} a sentence or a sequence of sentences \\
\textit{output:} prevailing sentiment of the input from categories: neutral, positive, negative.
\todo{doplnit diskuzi o ruzných možnostech definice}
\end{itemize}.

These tasks were chosen to show how pre-trained multilingual language models can help with different types of NLP tasks in one of trained langugages. 

\section*{Text structure}
Theoretical background in NLP and used AI methods and related work is provided in the \hyperref[chap:theandme]{following chapter}.

Implementation documentation in chapter \ref{chap:impl} is followed by discussion about used methods and experiment results in chapter \ref{chap:diss} Result models are accessible to user exploration as described in chapter \ref{chap:userdoc} and text is closed by \hyperref[chap:concl]{conclusion} with future work proposals.


