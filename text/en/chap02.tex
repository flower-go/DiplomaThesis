\chapter{Implementation analysis}
\label{chap:impl}
This chapter describes implemented linguistic models. As mentioned before, this work implements models for three Czech NLP tasks: tagging, lemmatization and sentiment analysis. Common model for first two tasks develops on the paper by \cite[]{straka2019czech} focused on application of contextual embeddings produced by language models as Bert %todo cite
 or Flair. %todo cite. 
 Third task, sentiment analysis, is perfomed by adding only one fully-connected layer at the top of bert model. So in the opposite to previous tasks, no sophisticated handcrafted pipeline is built and model relies only on the network powers.
Code for all models is available as an attachement of the thesis.
 \section{POS tagging and lemmatization model}
As mentioned earlier, model for this part is build upon a model (and a code) for previous work on Czech NLP processing with contextual embeddings \cite[]{straka2019czech}. Data preparation %todo trida s odkazem do github
pipeline - tokenization and sentence segmentation is taken over from the paper as well as base structure of lemmatizer and tagger network. 


%obrazek site - UDPipe 2.0 - vrstvy, optimizer
   %POS tagging a lemma jako classification (viz Milanův článek)
   %POS tagging a lemmata in one join network
   %embeddings - pretrained we - jaké to jsou?, trained we, character embeddings, bert jako prumerovani 
   %dictionary





%jak ten model vypada 
%bert embeddings - vdlijost
%bert model
%na cem se to evaluje - PDT popsat, sentment analysis data popsat


%obrazek berta
%implementace pokusneho kodu 
%použité knihovny - transformers



