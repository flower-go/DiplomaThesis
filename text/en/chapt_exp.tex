\chapter{Experiments}
\label{chap:exp}
This chapter describes all experiments and their results. The first part is dedicated to presentation of different experiement hyperparameters, which are in many cases common to all tasks, followed by the description of each task and a discussion of results.
\section{A description of training hyperparameters}
\label{sec:expe}
\subsection{General experiment setup (EXPE)}
Training is performed in one of the following settings:
\begin{itemize}
\item \textbf{base}: Baseline implementation (described separately for each task, typically without using advanced language models).
\item \textbf{ls}: This setup uses same setting as baseline implementation but with label smoothing.
\item \textbf{embed}: BERT-like language model is used only to generate static embeddings in advanced. These embeddings are not further trained.
\item \textbf{fine}: Fine-tuning consist of dividing the training time into two parts. Firstly, rest of the model is trained with BERT layers frozen (not trained), so it is same as the \textit{embed} settings. In the second part, the whole model is trained together.
\item \textbf{simple}: Model architecture is reduced to BERT layers with a simple classification head. This is a basic setting for all sentiment analysis experiments.\footnote{For tagging and lemmatization, all previously mentioned EXPE setups are performed with more sophisticated classification head than in \textit{simple} version.} 
\item \textbf{full}: This options means training the whole model from the beginning (in contrast to \textit{fine} option), but the classification head is not simplified (in contrast to \textit{simple} option).
\end{itemize}
\subsection{Training data}
Tagging and lemmatization tasks use the same set of data for every experiments, so there is no need for separate description. Sentiment analysis task, however, uses three possible options as a selection of training data:
\begin{itemize}
\item \textbf{mall|facebook|csfd}: Model is trained and evaluated on the (sub)set of Czech datasets.
\item \textbf{zero}: Model is trained on English sentiment analysis dataset, but evaluated on Czech data.
\item \textbf{eng}: Model is trained on the combination of Czech and English training data (and evaluated again on the Czech data). %TODO pridat jeste jednotlive ceske datasety
\end{itemize}
\subsection{Learning rate scheduling type (LRTYPE)}
Most experiments are expected to perform better with some kind of learning rate scheduling. This work implements three types of learning rate scheduling:
\begin{itemize}
\item \textbf{simple} \textit{Simple} option indicates no more complex learning rate scheduling than setting in advance different learning rates for different epochs.
\item \textbf{isrd} %TODO citovat 
\textit{isrd} means inverse square root learning rate decay defined by formula: $$1/\sqrt{n},$$  where $n$ is the current iteration.
\item \textbf{cos}: Another learning rate scheduling used in this work is \textit{cosine decay}, %TODO citovat
which applies the following formula: $$lr=lr_{min}^{i} + \frac{1}{2}\bigg(lr_{max}^{i} - lr_{min}^{i}\bigg)\bigg(1+\cos\bigg(\frac{T_{curr}}{T_i}\pi\bigg)\bigg),$$ where $lr_{min}^{i}$ and $lr_{min}^{i}$ is the range of the learning rate, $T_i$ is the number of epochs after which the learning rate is restarted, i.e. increased to the $lr_{max}^{i}$ value and $T_{curr}$ is the current epoch number.
\end{itemize}
Both \textit{cos} and \textit{isrd} are combined with \textit{warmup}. Learning rate is linearly increasing for first $k$ steps (one epoch in all experiments) from zero to the value in hyperparameters and than starts the decay.

\subsection{Model layers selected for embeddings (LAYERS)}
As discussed in the previous chapter, it is unclear how to extract best embeddings from the language model, especially which layers to take into account. According to the results published in \citep{Devlin2018} and \citep(Kondratyuk2019), we consider the following two promising approaches:
\begin{itemize}
\item \textbf{four}: Last four layers of the model are averaged to obtain final embeddings.
\item \textbf{att}: Layer attention performs weighted sum of all model layers, and the weights are trained during training together with the rest of the model.
\end{itemize}
Experiments are also performed with different learning rates (LR), batch size (BATCH), and a number of epochs (EPOCH). 
Technical details needed for running scripts with right arguments can be found in chapter \ref{chap:impl}.

\subsection{Metrics}
Metrics used for evaluation in this work are \textit{accuracy} and \textit{F1 score}. 
Accuracy is a percentage of correctly classified samples out of all samples, and it is the basic metric for all classification tasks (not only in this thesis). Accuracy can be sometimes misleading,\footnote{see: https://en.wikipedia.org/wiki/Accuracy\_paradox} and there exist other metrics which can better reflect experimenter's goals. One of them is F1 score, which is used together with accuracy for evaluation of sentiment analysis task due to comparability of results. F1 score is defined in terms of precision and recall. Precision  
$$\mathit{precision} = \frac{TP}{TP + FP}$$ express credibility of a positive result, e.g., if positive result means  a need of surgery, it is definitely unwanted to have low precision and perform many dangerous and expensive surgeries unnecessarily. Recall, defined as: $$\mathit{recall} = \frac{TP}{TP + FN},$$,
on the other hand tells us how many positives are captured. For example: \textit{How likely I am to be pregnant with a negative pregnancy test?} F1 score formula for binary classification is than defined as
$$F1 = 2 \dot \frac{precision \dot recall}{precision + recall}.$$
For multi-class classification, precision and recall needs to be redefined. F1 score can be computed per-class (for every class, binary classification of being in the class is taken). Per-class scores can be combined in one of following ways:
\begin{itemize}
\item macro-F1: average of per-class scores,
\item weighted-F1: average as before, but weighted by the number of samples in each class,
\item micro-F1: equals to accuracy.
\end{itemize}
\par
The following part describes specific details of each task.
\newpage
\section{Lemmatization and part-of-speech tagging}
\label{chap:tag}
Lemmatization and \acrlong{pos} tagging tasks are often categorized as morphological analysis, shares same architecture and trained network, so they will be described together in this section.
\subsection{Task Definition}
%co chci presne delat - vstup, vystup, metrika

\paragraph{\textbf{POS tagging}} \mbox{}\\
\textit{input}: a word \\
\textit{output}: tag, which contains not only part-of-speech (e.g. noun, pronoun, punctuation mark) but also other morphological analysis (case, tense, etc) corresponding to 15-places morphological tagging system by \cite{Hajic2004}. Description of each position can be found in Table \ref{Tab:tagset}.

\paragraph{\textbf{Lemmatization}} \mbox{}\\
\textit{input:} a word \\
\textit{output:} lemma -- a base form of a given words, for example nominative of singular for nouns or infinitive for verbs. In this work, lemmatization is treated as a classification problem with classes coresponding to generating rules which transform an input word into target lemma. For example of such rules see Figure \ref{fig:lemma_rules}. \\ %TODO kolik jich je v datasetu


Metric used for evaluation of the model is an accuracy reported separately for several options -- only tags/lemmas, accuracy of joint classification of tags and lemmas, and  also for all three variants with an usage of a morphological dictionary (this option is described in more detail in \ref{sub:dataset}).

\begin{figure}[H]
\centering
\includegraphics[width=1\textwidth]{../img/lemma_rules}
\protect\caption{
Table 1 from \citep{Straka2019b} presents 10 most common lemma generating rules in English EWT corpus. Each rule has two parts -- casing script for transforming uppercase and lowercase letters, and edit script. Edit script can transform prefix, suffix, or also a root of the word. %TODO trochu rozepsat a kde se da docist vice
}
\label{fig:lemma_rules}
\end{figure}

\begin{table}
\centering
\label{Tab:tagset}
\begin{tabular}{ |c|c|c| } 

 \hline
 Position & Name & Description \\ 
 \hline \hline
 1 & POS & Part of speech \\ \hline
 2 & SubPOS & Detailed part of speech \\ \hline
  3 & Gender & Gender \\ \hline
4 & Number & Number \\\hline
  5 & Case & Case \\ \hline
 6 & PossGender & Possessor's gender \\\hline
  7 & PossNumber & Possessor's number \\ \hline
8 & Person & Person \\\hline
  9 & Tense & Tense \\ \hline
 10 & Grade & Degree of comparison\\\hline
  11 & Negation & Negation \\ \hline
 12 & Voice & Voice \\\hline
 13 & Reserve1 & Reserve \\ \hline
14 & Reserve2 & Reserve \\\hline
  15 & Var & Variant, style \\ 
 \hline

\end{tabular}
\caption{Czech morphology developement is dated from 1989 \citep{Hajic2004} %TODO zdroj Hajič
and in description of words uses 15-places morphological tags as described in this table taken from https://ufal.mff.cuni.cz/pdt2.0/doc/manuals/en/m-layer/html/ch02s02s01.html}
\end{table}

\subsection{Related Work}
This work aims to improve previously published SOTA results for contextualized embeddings in czech lemmatization and tagging \citep{Straka2019}. 

as described in table \ref{Tab:tagset}.  %TODO a ja pouzivam ty samy? 


%\citep{Horsmann}
%\citep{Plank}
%\citep{Plisson}
%\citep{Straka2019b}
%\citep{Straka2019a}
%\citep{Toutanova2003}
%\citep{Wang2015}
%\citep{Huang2015}
%\citep{Collobert2011} ... preprocessing

\subsection{Dataset and Preprocessing}
\label{sub:dataset}
%TODO popisje v straka2019 - dopnit
Dataset for these tasks is taken from data of Prague Dependency Treebank (PDT) \citep{PDT35}, version 3.5 from year 2018. %TODO kolik tamjedat
Data consists of sentences with lemmas and tags. For ambiguous words, data contain all possible analysis. For example, Czech word "psa" have one possible lemma ("pes") but two possible tags because it could be one of two possible grammatical cases -- genitive or accusative. Input data for such word looks as follows: \\
\begin{center}
psa pes NNMS2-----A---- NNMS4-----A----
\end{center}.

Dataset was originally divided into tree parts - train, development and test, which is also used in this work. Input sentences are preprocessed as follows: %TODO mozna az do site? opsat vse z UDPipe2.0 !!!
\begin{itemize}
\item white space deletion
\item splitting into sentences and words
\item mapping characters and words into numbers -- mapping  words/characters which were found in train dataset into integers (from one to the number of unique words). This means that the network has no information about words/characters which appears only in test or development dataset. All newly appeared words/characters are mapped into one same number (typically $0$) for \textit{UNK} token/character.
\item tokenization -- Tokenizer for corresponding BERT-like model transforms input words into tokens. Each word is transformed into one or more strings, which are converted into numbers. This serves as an input into BERT part of model. To creating these input embeddings, the whole sentence for each word is needed as same words can have different representation in different contexts. More information can be found in \ref{sub:tokens}.
\end{itemize}

\subsection{Experiments and Architecture}

%For POS tagging, we applied a straightforward model in the lines of Ling et al. (2015) – first rep- resenting each word with its embedding, contextu- alizing them with bidirectional RNNs (Graves and Schmidhuber, 2005), and finally using a softmax classifier to predict the tags. z UDpipe2

%??? We perform tokenization, sentence segmentation and multi-word token splitting with the baseline UDPipe 1.2 approach. In a nutshell, input charac- ters are first embedded using trained embeddings, then fixed size input segments (of 50 characters) are processed by a bidirectional GRU (Cho et al., 2014), and each character is classified into three classes – a) there is a sentence break after this character, b) there is a token break after this char- acter, and c) there is no break after this character.

%reprezentation: tři typy embeddings - pretrained, trained, character-level a ještě berti

%TODO popisovat vice ty pravidla?

The model for lemmatization and tagging is build upon a model (and a code) for previous work on Czech NLP processing with contextual embedding \citep{straka2019czech}. 
Data preprocessing is taken over from the paper as well as the structure of a lemmatizer and a tagger network which is extended by BERT-like models, hoping for improvements. Previous work \citep{Straka2019} and \citep{Straka2018} showed that training tagging and lemmatization together in one network can be mutually advantageous, so both of these analysis are an output of one network and are trained jointly. Detailed visualisation of network architecture can be found in Figure \ref{pic:lt_arch}. \par The architecture of network can be divided into three parts -- inputs, optional \acrshort{rnn}s, classification head:
\paragraph{Inputs}
An input of the network is formed of five types -- characters (charseqs), words (charseq ids), correct responses(word ids), pretrained embeddings and possibly precomputed bert embeddings (depends on the experiment type). Before the further processing of inputs by \acrshort{rnn} cells, there are created two other types of embeddings: character-level embeddings and another word embeddings which are, in contrast to BERT and pretrained embeddings, also trained during training process.

\paragraph{RNN cells}
Characterlevel embeddings are further processed via \acrfull{gru} and all inputs (or their embeddings) are processed by recurrent part of network (specifically by \acrfull{lstm} cells).

\paragraph{Classification head(s)}
After the processing by recurrent neural networks, network employs two separate classification head, one for tagging and another for lemmatization. Both uses dense layer with tanh activation function to presented more non-linearity as used in \citep{2018} and a softmax function for obtaining the probability distribution over target classes. Lemmatization, however, presents one another change -- addition of character level data without RNN processing, which are used together with the rest of weights as an input into softmax following \citep{Straka2018}, because it leads to better performance of lemmatization in the case of shared network between both tasks.

\begin{figure}[ht]
\centering
\includegraphics[width=1\columnwidth]{../img/taggermodel.pdf}
\protect\caption{popis? }
\label{pic:lt_arch}
\end{figure}

\paragraph{Morphological Dictionary} All classification can be done with or without use of a morphological dictionary MorFlex \citep{11234/1-1834}, which can provide possible pairs \textit{tag-lemma}. If so, generated tag and lemma is a pair with maximal likelihood, but chosen just from the dictionary. This leads to more consistent results. 

\subsubsection{Experiments}
This part uses all main \textbf{experiment types} as decribed in \ref{sec:expe}: \textit{base, ls, embed, fine, simple, full}. Three \textbf{BERT-like models} are used for every experiment setup:
\begin{itemize}
\item multilingual BERT (mBERT) \citep{Devlin2019} 
\item XLM-RoBERTa \citep{Conneau2019}), 
\item RobeCzech \citep{Straka2021}.
\end{itemize}
XLM-RoBERTa and mBERT are trained on 100/104 different languages including Czech and RobeCzech is recently published version of RoBERTa, trained only on Czech data. %TODO nekde v diskuzi zminit i certa
\textbf{Selection of layers} are made in two ways -- last four layers and learning of weighted sum of all layers. These experiment are made for finetunning setup only and as the weighted sum does not showed a significant benefit, mean of last four layer is the only method used for other experiments. \textbf{Learning rate} is used as usual for each type of task a and three different learning schedules were applied in each combination of hyperparameters: cosine decay \textit{(cos)}, inverted square root decay \textit{(isrd)} and a one epoch warm-up followed by a constant learning rate \textit{(warmup)}. For \textit{embed} experiments, \textit{warmup} is replaced by  a simple division of training into tow parts with different learning rates as in %TODO citace. 
Batches has size 64, given by the compromise between the pursuit of relatively big batch size and computational resources.

%TODO popsat jak presne to vypada nebo alespon graficky ty learning rates
%classification head. %TODO popsat jak presne to vypada

\subsection{Results}
%Jsou vysledky lepsi nez baseline?
%Co je nejlepší jednotlivě a celkem?

%TODO tady upravit tabulka na nejlepší výsledky z kazde kategorie + baseline + porovnani
\begin{table}[!h]
  \begin{tabular}{|l||c|c|c||c|c|c|}
  \hline
\multirow{2}{*}{Experiment} & \multicolumn{3}{c||}{Without Dictionary}  &
      \multicolumn{3}{c|}{With Dictionary} \\ 
    & Tags & Lemmas & Both & Tags & Lemmas & Both \\ \hline
    StrakaB & 97.94\% & 98.75\% & 97.31\% & 98.05\% & 98.98\% & 97.65\% \\ \hline
    emb (lr 0.0001) &  97.80\% & 98.70\% & 97.17\% & 97.95\% & 98.93\% & 97.55\% \\ \hline
    baseline & 97.04 \% & 98.56 \% & 96.41\% &  97.31  \% & 98.83 \% & 96.90\% \\ \hline 
    StrakaC & 97.67\% & 98.63\% & 97.02\% & 97.91\% & 98.94\% & 97.51\% \\ \hline
  \end{tabular}
  \caption{%TODO cite
  Straka2019B is the best solution from \citep{Straka2019} paper. Straka2019C is a comparable solution  (BERT embeddings only), which was transformed into TF2 in this work.} 
\end{table}


\begin{table}[]
\begin{tabular}{lllllllllll}
\hline
   & Model       & EXPE  & LRTYPE         & LR                     & LemRaw & LemDict & TagsRaw & TagsDict & LemTagRaw & LemTagDict \\ \hline
0  & NA          & base  & simple         & 40:1e-3,20:1e-4        & 98,58  & 98,81   & 97,05   & 97,31    & 96,43     & 96,9       \\ \hline
1  & NA          & ls    & simple         & 40:1e-3,20:1e-4        & 98,55  & 98,81   & 97,12   & 97,34    & 96,51     & 96,94      \\ \hline
2  & mBERT       & embed & simple         & 40:1e-3,20:1e-4        & 98,69  & 98,93   & 97,83   & 97,98    & 97,17     & 97,58      \\ \hline
3  & mBERT       & embed & cos            & 60:1e-3                & 98,74  & 98,95   & 97,91   & 98,04    & 97,28     & 97,63      \\ \hline
4  & mBERT       & embed & isrd           & 60:1e-3                & 98,73  & 98,94   & 97,89   & 98,02    & 97,28     & 97,61      \\ \hline
5  & xlm-Roberta & embed & simple         & 40:1e-3,20:1e-4        & 98,57  & 98,8    & 97,33   & 97,54    & 96,68     & 97,12      \\ \hline
6  & xlm-Roberta & embed & cos            & 60:1e-3                & 98,6   & 98,83   & 97,45   & 97,62    & 96,81     & 97,21      \\ \hline
7  & xlm-Roberta & embed & isrd           & 60:1e-3                & 98,59  & 98,83   & 97,44   & 97,61    & 96,81     & 97,2       \\ \hline
8  & RoBECzech   & embed & simple         & 40:1e-3,20:1e-4        & 98,77  & 98,97   & 98,38   & 98,48    & 97,78     & 98,08      \\ \hline
9  & RoBECzech   & embed & cos            & 60:1e-3                & 98,79  & 98,99   & 98,38   & 98,48    & 97,8      & 98,1       \\ \hline
10 & RoBECzech   & embed & isrd           & 60:1e-3                & 98,78  & 98,98   & 98,4    & 98,48    & 97,8      & 98,09      \\ \hline
11 & mBERT       & fine  & simple         & 40:1e-3,20:1e-4,2:2e-5 & 98,69  & 98,93   & 97,84   & 97,99    & 97,21     & 97,59      \\ \hline
12 & fine        & cos   & 60:1e-3,5:3e-5 &                        & 98,72  & 98,95   & 97,97   & 98,08    & 97,33     & 97,68      \\ \hline
13 & fine        & isrd  & 60:1e-3,5:3e-5 &                        & 98,68  & 98,9    & 97,72   & 97,86    & 97,09     & 97,46      \\ \hline
14 & xlm-Roberta & fine  & simple         & 40:1e-3,20:1e-4,2:2e-5 & 98,62  & 98,84   & 97,72   & 97,9     & 97,07     & 97,48      \\ \hline
15 & fine        & cos   & 60:1e-3,5:3e-5 &                        & 98,67  & 98,9    & 97,95   & 98,09    & 97,32     & 97,69      \\ \hline
16 & fine        & isrd  & 60:1e-3,5:3e-5 &                        & 98,63  & 98,85   & 97,66   & 97,83    & 97,03     & 97,41      \\ \hline
17 & RoBECzech   & fine  & simple         & 40:1e-3,20:1e-4,2:2e-5 & 98,78  & 98,98   & 98,46   & 98,55    & 97,86     & 98,16      \\ \hline
18 & fine        & cos   & 60:1e-3,5:3e-5 &                        & 98,8   & 99      & 98,5    & 98,57    & 97,9      & 98,19      \\ \hline
19 & fine        & isrd  & 60:1e-3,5:3e-5 &                        & 98,76  & 98,95   & 98,33   & 98,41    & 97,72     & 98,02      \\ \hline
20 & mBERT       & fine  & simple         & 40:1e-3,20:1e-4,2:2e-5 & 98,67  & 98,91   & 97,76   & 97,92    & 97,13     & 97,52      \\ \hline
21 & fine        & cos   & 60:1e-3,5:3e-5 &                        & 98,72  & 98,95   & 97,98   & 98,1     & 97,34     & 97,69      \\ \hline
22 & fine        & isrd  & 60:1e-3,5:3e-5 &                        & 98,67  & 98,91   & 97,69   & 97,85    & 97,05     & 97,45      \\ \hline
23 & xlm-Roberta & fine  & simple         & 40:1e-3,20:1e-4,2:2e-5 & 98,6   & 98,81   & 97,62   & 97,77    & 96,96     & 97,35      \\ \hline
24 & fine        & cos   & 60:1e-3,5:3e-5 &                        & 98,67  & 98,89   & 97,91   & 98,06    & 97,29     & 97,66      \\ \hline
25 & fine        & isrd  & 60:1e-3,5:3e-5 &                        & 98,65  & 98,86   & 97,65   & 97,81    & 97,03     & 97,41      \\ \hline
26 & RoBECzech   & fine  & simple         & 40:1e-3,20:1e-4,2:2e-5 & 98,77  & 98,97   & 98,38   & 98,47    & 97,79     & 98,08      \\ \hline
27 & fine        & cos   & 60:1e-3,5:3e-5 &                        & 98,8   & 98,99   & 98,47   & 98,54    & 97,88     & 98,16      \\ \hline
28 & fine        & isrd  & 60:1e-3,5:3e-5 &                        & 98,77  & 98,96   & 98,33   & 98,41    & 97,72     & 98,01      \\ \hline
\end{tabular}
\label{tab:all_res_tl}
\caption{This table presents complete results for tagging and lemmatizationt tasks. }
\end{table}


\newpage
\section{Sentiment Analysis}
\label{chap:sent}
As stated in \citep{Veselovska}: "Sentiment analysis, also known as opinion mining, is an automatic detection of a positive or negative polarity, or neutrality of ... a text sequence", which is exactly as the sentiment analysis is understood in this work, although there are some other definitions consisting of e.g. opinion extraction, irony or stance \citep{Montoyo2012} and sentiment analysis can also continue with e.g., opinion extraction. Another tasks related to sentiment analysis is a subjectivity analysis (whether the presented opinion is objective or highly subjective), which is also not included in this work, mainly because the lack of labelled data for Czech. It is possible to analyze individual expressions, sentences or whole documents \citep{Veselovska}. This thesis focuses on the document-level classification, which has many real-life use cases and Czech training data are available.
\subsection{Task definition}
\paragraph{Sentiment analysis} \mbox{}\\
\textit{input:} sequence of sentences (a whole post or comment, depending on the source) \\
\textit{output:} prevailing sentiment of the input from categories: neutral, positive, negative.
\par


%TODO zero experiments
\paragraph{Metric} For evaluating performance, two metrics are used: weighted-F1 score and accuracy. Accuracy is a standard metric for classification and weighted-F1 allows better comparability and also provides additional insights into models evaluation.

\subsection{Related Work}
As every languge-related task, sentiment analysis is best explored for English. It is possible to derive sentiment by supervised learning (typical are Support Vector Machines or Maximum Entropy classifier) or using rule-based approach -- vocabulary of emotionally coloured words, emoticons etc \citep{Cano2019}, \citep{Veselovska}. BERT-like models were succesfully used to improve result for sentiment analysis task on English \citep{Devlin2019} and also other languages, for example Estonian \citep{Kittask2020}, Indonesian \citep{putra} or Italian \citep{pota2021effective}.
\par
There are not so many attempts to sentiment analysis in Czech in comparison to English, however some attempts were made with both neural networks and traditional machine learning - naive bayes classifiers, support vector machines, and maximum-entropy-based classifiers \citep{Veselovska}.  A thorough study of supervised machine learning methods on \textit{mall} and \textit{facebook} dataset is offered in \citep{Cano2019}. For a practical use, \citep{Zizka} presented automatic sentiment prediction of unlabelled text based on a small set of labelled patterns via searching similarities. As the neural networks dominates in many NLP taks, they are also applied in sentiment analysis. One of first attepts to apply neural networks on Czech sentiment is described in \citep{Lenc}, which evaluates besides others all three datasets used in this work on document-level sentiment analysis. \citep{kysely} performes sentiment analysis using embeddings and convolutional neural network on multidimensional embedding, which is quite unusual as CNNs are typically used for image processing. \citep{kysely} uses same three datasets, but classifies only on sentence-level (they filtered out longer samples), which is simpler as longer texts tend to be more inconsistent about sentiment \citep{Veselovska}. \citep{Libovicky} presents state-of-the art results in three czech NLP tasks including sentiment analysis. They use only CSFD dataset with resulting accuracy 80.8\% which is comparable to previous \acrshort{sota} \citep{Brychcin2013}. The second mentioned paper uses quite complicated method for classification incorporating the fact of which movie is reviewed, while \citep{Libovicky} uses only bidirectional LSTMs with multiple attention heads following state-of-the-art results on English \citep{Lin2017}. There are five previous works know to me, which involves BERT-like models in Czech sentiment: 
\begin{itemize}
\item XLM-Roberta applied on all three datastets trimmed to 128 characters \footnote{http://www.janpalasek.com/sentiment-analysis-czech.html}
\item \citep{Klouda} applies multilingual BERT on mall dataset with resulting accuracy about  81\%, which did not outperform the naive bayes classifier baseline with 84\% accuracy, 
\item  \citep{Sido2021} presents monolingual Czech model Czert,  based on BERT and ALBERT models, and evalutes it on csfd and facebook datastets with new state-of-the-art results,
\item \citep{Straka2021} published another monolingual model, based on more succesfull RoBERTa model and surpassed Czert on facebook dataset.
\end{itemize}

\subsection{Dataset and Preprocessing}
%TODO habernal vytvoril ty datasty 
%TODO zjistit jaka jsou na to porovnavaci data This work primary tries to improve existing tasks and show the ability of contextualized embeddings to improve results, so tasks with existing data and results were selected.  %TODO why?
Four main Czech datasets with sentiment annotation are available: news from Aktualne.cz (aktualne) \citep{Veselovska}, user reviews from MALL.cz (mall), film reviews from csfd.cz (csfd), and posts from Czech branch pages on facebook.cz (facebook) (last tree in \citep{Habernal.et.al.2013}). As \textit{aktualne} dataset turned out to be problematical because the text were ambiguous even for annotators, and its authors later used other mentioned datasets \citep{Veselovska}, this work also focuses only on the three other data sources -- \textit{mall}, \textit{csfd} and \textit{facebook} \footnote{All three datasets are all available here: http://liks.fav.zcu.cz/sentiment/}. Some experiment are also performed with in-domain training on English data. For this purpose is used \textit{imdb} dataset, \footnote{https://www.tensorflow.org/datasets/catalog/imdb\_reviews} which contains movie reviews from the biggest movie rating website imdb.com. This leads to some problems described later in this section, because English dataset contains only binary classification (positive/negative). Table \ref{tab:datasets} summarize each dataset. All dataset were randomly split into train, development and test datasets with the same labels distribution as original datasets.
\par
\begin{figure}[!h]
\centering
\includegraphics[width=0.95\columnwidth]{../img/dist_all.png}
\protect\caption{Distribution of positive/neutral/negative labels in each dataset.}
\label{pic:dist}
\end{figure}

\begin{figure}[!h]
\centering
\includegraphics[width=0.65\columnwidth]{../img/all.png}
\protect\caption{Percentage and absolute values of labels in all three Czech datasets together.}
\label{pic:dist_all}
\end{figure}
As can be seen in figure \ref{pic:dist}, distribution of labels differs among datasets. Moreover, Figure \ref{pic:dist_all} shows, that the resulting dataset is highly unbalanced, which may causes divergence and stuck in training. Due to the big part of labels being positive, many learning strategies just ended with predicting only \textit{positive} class, i.e. 55\% accuracy, so unfortunately learned nothing. 
\begin{center}
\begin{table}[!h]
\begin{tabular}{|l||lll|}
\hline
     & length & labels                                                                & domain        \\ \hline \hline
\textbf{mall}     & 145306 & \begin{tabular}[c]{@{}l@{}}positive\\ neutral\\ negative\end{tabular} & domestic appliance reviews                             \\ \hline
\textbf{csfd} & 91304  & \begin{tabular}[c]{@{}l@{}}positive\\ neutral\\ negative\end{tabular} & movie reviews \\ \hline
\textbf{facebook} & 9752   & \begin{tabular}[c]{@{}l@{}}positive\\ neutral\\ negative\end{tabular} & \begin{tabular}[c]{@{}l@{}}brand pages of e.g. shops or mobile network\\ providers\end{tabular} \\ \hline
\textbf{imdb} & 25000  & \begin{tabular}[c]{@{}l@{}}positive\\ negative\end{tabular}           & movie reviews \\ \hline
\end{tabular}
\caption{Three Czech datasets (mall, facebook, csfd) and one English (imdb) are used for training in this work.}
\label{tab:datasets}
\end{table}
\end{center}

\subsection{Experiments and Architecture}
The main division of experiments is by the input dataset -- each of Czech models separately and one joint dataset consisting of all Czech datasets, i.e. four different datasets. All variants performed both layers attention and an average of last four layers. As for the learning rate, all experiments were made with learning rate $3 \times 10^{-5}$ and there were always two types of learning rate decay -- cosine and inverse square root decay. 
\par
Network architecture is much simpler than in tagging and lemmatization task and corresponds to \textit{simple} setting of these tasks -- only BERT-like model and classification head consisting of layer with softmax activation function. 
\par 
Baseline for these models is Naive Bayes Classifer (NB) with term frequency--inverse document frequency (tf--idf) representation. \textit{tf\_idf}-s for each word are defined in this way: \textit{tf} stands for a term frequency
\[ tf = \frac{word\_occurences}{number\_of\_words\_in\_document} \]
and this is count over the whole dataset, while \textit{idf} is inverse document frequency
\[idf = \frac{number\_of\_documents}{number\_of\_documents\_with\_word}.\]\textit{Idf} works as an evaluation of the importance of the word. The result is then: \[tf\_idf = tf \cdot idf .\] 
This representation serves as an input into Naive Bayes Classifier (NB). NB \citep{duda1973pattern} is a probabilistic model, which models probability of the class $k$ given the data features $x_i$: $p(C_k|x_1,...,x_n)$ and uses \textit{naive} assumption of features to be independent. 

\subsection{Results and Discussion}
Models based upon RoBECzech not only outperformed baseline, but also achieved new state-of-the-art results in all three datasets as can be seen in table \ref{tab:res_sent_best}. Complete results can be seen in table \ref{tab:res_all_sent}. The best model for each dataset was the one trained on that dataset, although model trained on joint datasets performed comparable to single-data model on \textit{csfd} and \textit{mall}. For \textit{facebook}, joint model was worse by 9\%. This can be caused by the a difference in distribution of labels between datasets. \textit{Facebook} dataset has the most different label distribution in comparison to \textit{all\_czech} together (mostly neutral posts vs. mostly positive, see Fig. \ref{pic:dist} and Fig. \ref{pic:dist_all}). 

%TODO sjednotit jmeno spolecneho datasetu

%TODO chyba v predikci, prohozeni
%mall: outputs/0608-2341_sent_o_63 , 'sentiment_analysis.py-2021-06-08_234151-a=12,bs=4,b=..'
%facebook: 'sentiment_analysis.py-2021-06-08_172844-a=12,bs=4,b=..'
%csfd: sentiment_analysis.py-2021-07-05_115521-a=16,bs=2,b=..
%bezi mi join

%TODO pusten mall

%TODO napsat proc to neni crossvalidovane a ze ty cisla tedy jsou divna, pusteno s tremi ruznymi seedy tak pak udelam prumer
%TODO 16 pusteno na 10 epoch - stale bezi, stale bezi
%TODO habernal limity accuracy a oduvodneni v kysely



\begin{table}[!h]
\centering
\begin{tabular}{|l|l||ll|}
\hline
dataset                    & models      & Acc   & F1    \\ \hline \hline
\multirow{5}{*}{All Czech} & baseline    & 82.00 & 70.00 \\ \cline{2-4} 
                           
                           
                           
                           & \textit{\citep{kysely}} & \textit{67.82} & \textit{67.00} \\ \cline{2-4}
                           & best(16)    & \textbf{84.04} & \textbf{84.86} \\ \hline \hline
\multirow{3}{*}{csfd}      & baseline    & 69.07 & 69.00 \\ \cline{2-4} 
& \textit{Czert}       &       & \textit{84.79} \\ \cline{2-4} 
& $\textit{\citep{kysely}}\star$ & \textit{71.34} & \textit{71.00} \\ \cline{2-4}
                           & best(16)    & 84.02 & 84.00\\ \cline{2-4} 
                           & best(69)    & \textbf{84.89 }& \textbf{84.87} \\ \hline \hline
\multirow{3}{*}{mall}      & baseline    & 84.72 & 83.00 \\ \cline{2-4} 
& \textit{\citep{kysely}} & \textit{82.52} & \textit{81.00} \\ \cline{2-4}
& \textit{\citep{Klouda}} & 81.00 & 79.00 \\ \cline{2-4}
                           & best(16)    & 84.40 & 84.00 \\ \cline{2-4} 
                           & best(63)    & \textbf{84.60} & \textbf{84.14} \\ \hline \hline
\multirow{3}{*}{facebook}  & baseline    & 67.30 & 63.00 \\ \cline{2-4} 
& \textit{RobeCzech }  &       & \textit{80.13} \\ \cline{2-4} 
& \textit{XLM-RoBERTa} &       & \textit{82.29} \\ \cline{2-4} 
& \textit{Czert}       &       & \textit{76.55} \\ \cline{2-4} 
& \textit{\citep{kysely}} & \textit{71.62} & \textit{71.00} \\ \cline{2-4}
                           & best(16)    & 75.00 & 74.98 \\ \cline{2-4} 
                           & best(45)    & \textbf{81.80} & \textbf{81.65} \\ \hline
\end{tabular}
\caption{Best results for all datasets and a comparison to previous work. Best(16) is a best model for joint dataset and best(x) is always the best model for respective dataset. Numbers in italics are from related work. \protected\\ $\star$ \citep{kysely} performs only sentence-level classification. }
\label{tab:res_sent_best}
\end{table}

\begin{table}[!h]
\centering
\begin{tabular}{|l||c|c|c|c|}
\hline
 Input &
  joint &
  mall &
  facebook & csfd \\ \hline \hline
\begin{tabular}[c]{@{}l@{}}Rozbila se po prvním použití, je na hovno.\\ \textit{It broke after the first use, it is shitty.}\end{tabular} &
  \cellcolor[HTML]{648FFF}Neg &
  \cellcolor[HTML]{648FFF}Neg &
  \cellcolor[HTML]{648FFF}Neg & 
  \cellcolor[HTML]{DC267F}Neut\\ \hline
\begin{tabular}[c]{@{}l@{}}Rozbila se až za rok.\\ \textit{It broke after a year of use.}\end{tabular} &
  \cellcolor[HTML]{648FFF}Neg &
  \cellcolor[HTML]{648FFF}Neg &
  \cellcolor[HTML]{648FFF}Neg & 
  \cellcolor[HTML]{DC267F}Neut\\ \hline
\begin{tabular}[c]{@{}l@{}}S manželem jsme si víkend moc užili.\\ \textit{Me and my husband enjoyed the weekend.}\end{tabular} &
  \cellcolor[HTML]{FFB000}Pos &
  \cellcolor[HTML]{FFB000}Pos &
  \cellcolor[HTML]{FFB000}Pos & 
  \cellcolor[HTML]{FFB000}Pos \\ \hline
\begin{tabular}[c]{@{}l@{}}Ok, ale nic zajímavého.\\ \textit{Ok, but nothing interesting}.\end{tabular} &
  \cellcolor[HTML]{DC267F}Neut &
  \cellcolor[HTML]{648FFF}Neg &
  \cellcolor[HTML]{648FFF}Neg &
   \cellcolor[HTML]{648FFF}Neg \\ \hline
\begin{tabular}[c]{@{}l@{}}super zboží \\ \textit{super product}.\end{tabular} &
  \cellcolor[HTML]{FFB000}Pos &
  \cellcolor[HTML]{FFB000}Pos &
  \cellcolor[HTML]{FFB000}Pos &
  \cellcolor[HTML]{DC267F}Neut \\ \hline
\end{tabular}
\caption{Evaluation of models on four Czech sentences. \textit{mall} model was not included as a separate option, as the best model is the joint one.}
\label{tab:four_sent}
\end{table}
%TODO oznacit sotas nejakym symbolem
Following \citep{kysely}, resulting models were evaluated on five different czech sentences to manifest the differences between models (Table \ref{tab:four_sent}). It can be seen that predicting neutral vs. negative is still tricky for models, which can be also seen in confusion matrices \ref{tab:conf}. Confusion matrices shows, that predicting neutral is complicated in general, meanwhile models have learned to distinguish well between positive and negative sentiment.  Table \ref{tab:four_sent} also shows that \textit{csfd} model is quite different from the rest. It is probably caused by the difference in the training data nature. 

\begin{table}[!h]
\centering
\begin{tabular}{|llllll||llll|}
\hline
\multicolumn{5}{|c}{Combined datasets (16)}                          &  & \multicolumn{4}{c|}{mall (63)}        \\
                             & \multicolumn{4}{c}{predicted labels} &  & \multicolumn{4}{c|}{predicted labels} \\
\multirow{4}{*}{\rotatebox[origin=c]{90}{True labels}} &         & neut    & neg    & pos     &  &         & neut    & neg     & pos   \\
                             & neut    & 7027    & 885    & 2025    &  & neut    & 2896    & 199     & 1696   \\
                             & neg     & 1122    & 4783   & 307     &  & neg    & 344     & 1082    & 132    \\
                             & pos     & 1279    & 207    & 18857   &  & pos     & 932     & 54      & 14461  \\ &&&&&&&&&\\ \hline \hline
\multicolumn{5}{|c}{csfd (69)}                                       &  & \multicolumn{4}{c|}{facebook (45)}    \\
\multirow{4}{*}{\rotatebox[origin=c]{90}{True labels}} &         & neut    & neg    & pos     &  &         & neut    & neg      & pos   \\
                             & neut    & 3600    & 685    & 170     &  & neut    & 440     & 41       & 51    \\
                             & neg     & 603     & 3769   & 241     &  & neg     & 57      & 135      & 7     \\
                             & pos     & 146     & 225    & 4257    &  & pos     & 23      & 3        & 243  \\ &&&&&&&&& \\ \hline
\end{tabular}
\caption{Confusion matrices for best model in each category.}
\label{tab:conf}
\end{table}


%TODO tady mam best pro ruzne datasety a neni to uplne porovnatene

% Please add the following required packages to your document preamble:
% \usepackage{multirow}
\begin{table}[]
\centering
\resizebox*{!}{\textheight-2pt}{\begin{tabular}{|l|l|l|l||ll|}
\hline
\multicolumn{2}{|l|}{MODEL}       & EXPE                      & LRTYPE                & Acc    & F1   \\ \hline  \hline
1  & \multirow{6}{*}{mBERT}     & czech                     & \multirow{3}{*}{Isrd} & 80.89   & 80.62 \\ \cline{1-1} \cline{3-3} \cline{5-6}
2  &                            & zero                      &                       & 49.51   & 44.67 \\ \cline{1-1} \cline{3-3} \cline{5-6}
3  &                            & eng                       &                       & 81.17   & 80.90 \\ \cline{1-1} \cline{3-6}
4  &                            & czech                     & \multirow{3}{*}{cos}  & 82.56   & 82.35 \\ \cline{1-1} \cline{3-3} \cline{5-6}
5  &                            & zero                      &                       & 53.41   & 47.64 \\ \cline{1-1} \cline{3-3} \cline{5-6}
6  &                            & eng                       &                       & 82.55   & 82.37 \\ \hline
13 & \multirow{4}{*}{RoBECzech} & czech                     & \multirow{2}{*}{isrd} & 81.17   & 80.90 \\ \cline{1-1} \cline{3-3} \cline{5-6}
14 &                            & zero                      &                       & 55.31   & 48.26 \\ \cline{1-1} \cline{3-6}
16 &                            & czech                     & \multirow{2}{*}{cos}  & 84.04   & 83.86 \\ \cline{1-1} \cline{3-3} \cline{5-6}
17 &                            & zero                      &                       & 57.64   & 48.79 \\ \hline
19 & \multirow{6}{*}{mBERT}     & czech                     & \multirow{3}{*}{Isrd} & 81.61   & 81.43 \\ \cline{1-1} \cline{3-3} \cline{5-6}
20 &                            & zero                      &                       & 53.92   & 47.55 \\ \cline{1-1} \cline{3-3} \cline{5-6}
21 &                            & eng                       &                       & 81.79   & 81.32 \\ \cline{1-1} \cline{3-6}
22 &                            & czech                     & \multirow{3}{*}{cos}  & 82.62   & 82.42 \\ \cline{1-1} \cline{3-3} \cline{5-6}
23 &                            & zero                      &                       & 51.99   & 46.63 \\ \cline{1-1} \cline{3-3} \cline{5-6}
24 &                            & eng                       &                       & 82.59   & 82.36 \\ \hline
31 & \multirow{4}{*}{RoBECzech} & czech                     & \multirow{2}{*}{isrd} & 83.26   & 83.18 \\ \cline{1-1} \cline{3-3} \cline{5-6}
32 &                            & zero                      &                       & 58.36   & 50.40 \\ \cline{1-1} \cline{3-6}
34 &                            & czech                     & \multirow{2}{*}{cos}  & 83.88   & 83.68 \\ \cline{1-1} \cline{3-3} \cline{5-6}
35 &                            & zero                      &                       & 58.13   & 50.89 \\ \hline
37 & \multirow{2}{*}{mBERT}     & \multirow{7}{*}{facebook} & Isrd                  & 75.30   & 74.97 \\ \cline{1-1} \cline{4-6}
38 &                            &                           & cos                   & 76.20   & 75.89 \\ \cline{1-2} \cline{4-6}
41 & \multirow{5}{*}{RoBECzech} &                           & isrd                  & 80.10   & 79.87 \\ \cline{1-1} \cline{4-6}
42 &                            &                           & simple                & 79.20   & 79.12 \\ \cline{1-1} \cline{4-6}
43 &                            &                           & \multirow{3}{*}{cos}  & 81.50   & 81.37 \\ \cline{1-1} \cline{5-6}
44 &                            &                           &                       & 81.00   & 80.78 \\ \cline{1-1} \cline{5-6}
45 &                            &                           &                       & 81.80   & 81.65 \\ \hline
46 & \multirow{2}{*}{mBERT}     & \multirow{4}{*}{facebook} & Isrd                  & 76.40   & 75.67 \\ \cline{1-1} \cline{4-6}
47 &                            &                           & cos                   & 77.20   & 76.83 \\ \cline{1-2} \cline{4-6}
50 & \multirow{2}{*}{RoBECzech} &                           & isrd                  & 79.60   & 79.07 \\ \cline{1-1} \cline{4-6}
51 &                            &                           & cos                   & 80.60   & 80.38 \\ \hline
52 & \multirow{2}{*}{mBERT}     & \multirow{8}{*}{mall}     & Isrd                  & 82.80   & 82.80 \\ \cline{1-1} \cline{4-6}
53 &                            &                           & cos                   & 84.27   & 83.88 \\ \cline{1-2} \cline{4-6}
56 & \multirow{2}{*}{RoBECzech} &                           & isrd                  & 83.17   & 83.37 \\ \cline{1-1} \cline{4-6}
57 &                            &                           & cos                   & 84.73   & 84.30 \\ \cline{1-2} \cline{4-6}
58 & \multirow{2}{*}{mBERT}     &                           & Isrd                  & 83.02   & 82.90 \\ \cline{1-1} \cline{4-6}
59 &                            &                           & cos                   & 84.04   & 83.61 \\ \cline{1-2} \cline{4-6}
62 & \multirow{2}{*}{RoBECzech} &                           & isrd                  & 84.08   & 83.88 \\ \cline{1-1} \cline{4-6}
63 &                            &                           & cos                   & 84.60   & 84.14 \\ \hline
64 & \multirow{2}{*}{mBERT}     & \multirow{8}{*}{csfd}     & Isrd                  & 80.77   & 80.83 \\ \cline{1-1} \cline{4-6}
65 &                            &                           & cos                   & 82.04   & 82.04 \\ \cline{1-2} \cline{4-6}
68 & \multirow{2}{*}{RoBECzech} &                           & isrd                  & 83.06   & 83.05 \\ \cline{1-1} \cline{4-6}
69 &                            &                           & cos                   & 84.89   & 84.87 \\ \cline{1-2} \cline{4-6}
70 & \multirow{2}{*}{mBERT}     &                           & Isrd                  & 81.63   & 81.60 \\ \cline{1-1} \cline{4-6}
71 &                            &                           & cos                   & 82.20   & 82.19 \\ \cline{1-2} \cline{4-6}
74 & \multirow{2}{*}{RoBECzech} &                           & isrd                  & 83.13   & 83.18 \\ \cline{1-1} \cline{4-6}
75 &                            &                           & cos                   & 84.32   & 84.32 \\ \hline
\end{tabular}}
\caption{This table presents complete results on the sentiment task. }
\label{tab:res_all_sent}
\end{table}

%Because BERT model was trained on multilingual data, it is naturally not so good in language minoritly presented in the Bert's training data. When transfering the learned knowledge to Czech sentiment task, we actually want to improve model in two ways: teach it something more specific about given task, i.e., sentiment, and improve its knowledge about selected language (czech in this case). By using czech sentiment dataset, both thing are incorporated into training. To obtained better results and following the \citep{Putra}, I also selected english sentiment dataset. The idea behind is that BERT is quite good in english and maybe can learn faster useful knowledge about the given task from data in more familiar language.


%results of zero2 without training
%[[ 2809  6986   142]
 %[ 1538  4621    53]
 %[ 4787 15202   354]]
%Test accuracy: 0.21330702619752273
%F1 metrics: 0.1467527264529802

%jeste pusteno 35 to je s robeckem, jestli neni na tu cestinu prior lepsi
 %Test accuracy: 0.3478296612956264
%F1 metrics: 0.3179712252536629
 


















