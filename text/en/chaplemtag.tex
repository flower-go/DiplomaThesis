\chapter{Task 1: Lemmatization and part-of-speech tagging}
\label{chap:tag}
%TODO definice
Czech morphology developement is dated from 1989 %TODO zdroj Hajič.
and in description of words uses 15-places morphological tags \citep{Hana2005}.
A more detailed description of each position can be seen in table \ref{Tab:tagset}.

\begin{table}
\centering
\label{Tab:tagset}
\begin{tabular}{ |c|c|c| } 


 \hline
 Position & Name & Description \\ 
 \hline \hline
 1 & POS & Part of speech \\ \hline
 2 & SubPOS & Detailed part of speech \\ \hline
  3 & Gender & Gender \\ \hline
4 & Number & Number \\\hline
  5 & Case & Case \\ \hline
 6 & PossGender & Possessor's gender \\\hline
  7 & PossNumber & Possessor's number \\ \hline
8 & Person & Person \\\hline
  9 & Tense & Tense \\ \hline
 10 & Grade & Degree of comparison\\\hline
  11 & Negation & Negation \\ \hline
 12 & Voice & Voice \\\hline
 13 & Reserve1 & Reserve \\ \hline
14 & Reserve2 & Reserve \\\hline
  15 & Var & Variant, style \\ 
 \hline

\end{tabular}
\caption{Table from \citep{Hana2005} describes all 15 positions of Czech morphological tagging.
} 
\end{table}
%TODO related work
%TODO sota
This work aims to improve previously published SOTA results for contextualized embeddings in czech lemmatization and tagging \citep{Straka2019}. 

\begin{table}[!h]
  \begin{tabular}{|l||c|c|c||c|c|c|}
  \hline
\multirow{2}{*}{Experiment} & \multicolumn{3}{c||}{Without Dictionary}  &
      \multicolumn{3}{c|}{With Dictionary} \\ 
    & Tags & Lemmas & Both & Tags & Lemmas & Both \\ \hline
    StrakaB & 97.94\% & 98.75\% & 97.31\% & 98.05\% & 98.98\% & 97.65\% \\ \hline
    emb (lr 0.0001) &  97.80\% & 98.70\% & 97.17\% & 97.95\% & 98.93\% & 97.55\% \\ \hline
    baseline & 97.04 \% & 98.56 \% & 96.41\% &  97.31  \% & 98.83 \% & 96.90\% \\ \hline 
    StrakaC & 97.67\% & 98.63\% & 97.02\% & 97.91\% & 98.94\% & 97.51\% \\ \hline
  \end{tabular}
  \caption{%TODO cite
  Straka2019B is the best solution from \citep{Straka2019} paper. Straka2019C is a comparable solution  (BERT embeddings only), which was transformed into TF2 in this work.} 
\end{table}
%TODO jak jsem to udelala
%TODO  vysledky

\citep{Horsmann}
\citep{Plank}
\citep{Plisson}
\citep{Straka2019b}
\citep{Straka2019a}
\citep{Toutanova2003}
\citep{Wang2015}


